\documentclass[a4paper, 12pt]{report}
\usepackage[utf8]{inputenc}
\usepackage{tikz}
\usepackage{graphicx}
\usepackage{geometry}
\usetikzlibrary{calc,patterns,angles,quotes}
\geometry{margin=1in,left=0.6in,right=0.6in,bottom=0.5in}
\usepackage{fancyhdr}
\usepackage{hyperref}
\usepackage{chemfig}
\usetikzlibrary{calc}
\usepackage{enumitem}
\usepackage{placeins}
\usepackage{caption}
\usepackage{array}
\usepackage{float}
\usepackage{amsmath, amssymb, amscd, MnSymbol,mathrsfs}
\usepackage{tcolorbox}
\usepackage{bibref}
\usepackage{bm}
\newcommand{\vect}[1]{\boldsymbol{\mathbf{#1}}}
\usepackage{pdfpages}

\usepackage{empheq}
\usepackage{pgfplots}
\pgfplotsset{compat=1.18}
\usetikzlibrary{calc, angles, quotes}
\usepackage[oldvoltagedirection]{circuitikz}
\usepackage{booktabs}
\usepackage{array}
\usepackage{arydshln}
\usepackage{tcolorbox}

\usepackage{hyperref}
\usepackage{cancel}
\usepackage{changepage}
\usepackage{placeins}
\usepackage{enumitem}
\usepackage{pgf}
\usetikzlibrary{decorations.markings}
\usetikzlibrary{decorations.pathmorphing}
\usepackage{tikz-3dplot}

\renewcommand{\thechapter}{\Alph{chapter}} % Use letters for parts
\renewcommand{\chaptername}{Part} % Change "Chapter" to "Part"
\renewcommand{\thesection}{\Alph{chapter}.\arabic{section}} % Use letters for sections

\usepackage{listings}
\usepackage{bm}
\usepackage{minted}

\usepackage{courier}
\usepackage{xcolor}
\usepackage{color}
\def\ni{green!60!black!40!white}

\lstdefinestyle{mypythonstyle}{
    language=Python,
    basicstyle=\footnotesize\ttfamily,
    numbers=left,
    numberstyle=\tiny,
    numbersep=9pt,
    showstringspaces=false,
    frame=single,
    breaklines=true,
    keywordstyle=\color{blue!80!black}\bfseries,  % Darker blue and bold for keywords
    commentstyle=\color{green!60!black},  % Softer gray for comments
    stringstyle=\color{orange},  % Different color for strings
    backgroundcolor=\color{gray!5},  % Light gray for background
    tabsize=4,
    belowcaptionskip=10pt, % Adjust space below caption
    morecomment=[l]{//},  % For inline comments
    literate={~} {$\sim$}{1},  % Converts tilde to a symbol
}

\lstdefinestyle{matlab}{
    language=MATLAB,
    basicstyle=\ttfamily\footnotesize,
    breaklines=true,
    commentstyle=\color{green!60!black},
    keywordstyle=\color{blue!80!black}\bfseries,
    stringstyle=\color{red!70!black},
    numberstyle=\tiny\color{gray},
    frame=single,
    framesep=5pt,
    rulecolor=\color{black!30},
    backgroundcolor=\color{white!94!gray},
    emphstyle=\color{purple}\bfseries,
    emph={[2]\%},
    emphstyle={[2]\color{green!60!black}},
    showstringspaces=false,
    tabsize=4,
    belowcaptionskip=10pt, % Adjust space below caption
    morekeywords={import,classdef,properties,methods,end,
        function,return,if,else,elseif,switch,case,otherwise,
        for,while,break,continue,try,catch,global,persistent},
    morecomment=[l][\color{magenta!70!black}]{\%\%},
    morestring=[b]',
    literate=
    *{0}{{{\color{cyan!70!black}0}}}1
    {1}{{{\color{cyan!70!black}1}}}1
    {2}{{{\color{cyan!70!black}2}}}1
    {3}{{{\color{cyan!70!black}3}}}1
    {4}{{{\color{cyan!70!black}4}}}1
    {5}{{{\color{cyan!70!black}5}}}1
    {6}{{{\color{cyan!70!black}6}}}1
    {7}{{{\color{cyan!70!black}7}}}1
    {8}{{{\color{cyan!70!black}8}}}1
    {9}{{{\color{cyan!70!black}9}}}1
}

\def\link{blue!50!black}
\makeatletter
\renewcommand*\env@matrix[1][\arraystretch]{%
    \edef\arraystretch{#1}%
    \hskip -\arraycolsep
    \let\@ifnextchar\new@ifnextchar
    \array{*{\c@MaxMatrixCols}c}
}
\makeatother


\pagestyle{fancy}
\fancyhf{}
\fancyhead[L]{Module: EG4017 - Engineering Mathematics}
\fancyhead[R]{Page \thepage}

\title{\vspace{3em} \Huge \textbf{Engineering\\ Mathematics and Computing}\\ \vspace{1em} \Large Task 2: Coursework Assessment}
\author{Student Name: Sakariye Abiikar\\ KID: 2371673}
\date{Last Updated: November 6, 2024\\ Submission Deadline: December 12, 2024, 5pm \\[1em] Git Repo : \color{blue}\url{https://github.com/sakx7/mathcompuni2}}

\begin{document}
    
    \maketitle
    \thispagestyle{empty}
    
    \newpage
    \thispagestyle{empty}
    \newgeometry{margin=1in,left=0.6in,right=0.6in,bottom=0in}
    
    \chapter{Mathematics}
    
    \newpage\centering\restoregeometry
    
    \setcounter{page}{1}
    \fancyhead[L]{Part A: Mathematics}
    
    
    \begin{tcolorbox}[title={\color{black}\section{Q1}}, colback=white, colframe=\ni, boxrule=1mm, width=1\textwidth]
        1. Find \( \int \frac{1}{7x + 6} \, dx \)
    \end{tcolorbox}
    \[\int \frac{1}{7x + 6} \, dx\]
    Use substitution, Let 
    \[u = 7x + 6\]
    \[\frac{du}{dx} = 7 \quad \Rightarrow \quad dx = \frac{du}{7}\]
    Substituting into the integral, we get
    \[\int \frac{1}{7x + 6} \, dx = \frac{1}{7} \int \frac{1}{u} \, du\]
    This is a standard integral of the reciprocal
    \[\frac{1}{7} \int \frac{1}{u} \, du = \frac{1}{7} \ln |u| + C\]
    Finally, substituting back \( u = 7x + 6 \), we get
    \[\boxed{\int \frac{1}{7x + 6} \, dx = \frac{1}{7} \ln |7x + 6| + C}\]
    
    
    \newpage
    
    \begin{tcolorbox}[title={\color{black}\section{Q2}}, colback=white, colframe=\ni, boxrule=1mm, width=1\textwidth]
        2. Find \( \int \frac{x}{\sqrt{4 - x^2}} \, dx \)
    \end{tcolorbox}
    
    \[\int \frac{x}{\sqrt{4 - x^2}} \, dx\]
    Use substitution, Let\\[-1.5em] 
    \[u = 4 - x^2\]
    \[\frac{du}{dx} = -2x \quad \Rightarrow \quad dx = -\frac{du}{2x}\]
    Substituting into integral, we get
    \[\int \frac{x}{\sqrt{4 - x^2}} \, dx = \int \frac{x}{\sqrt{u}} \cdot \left(-\frac{du}{2x}\right) = -\frac{1}{2} \int \frac{1}{\sqrt{u}} \, du\]
    This is a standard integral of the exponent/power
    \[-\frac{1}{2} \int u^{-\frac{1}{2}} \, du = -\frac{1}{2} \cdot 2 \sqrt{u} + C = -\sqrt{u} + C\]
    Finally, substituting back \( u = 4 - x^2 \), we get
    \[\boxed{\int \frac{x}{\sqrt{4 - x^2}} \, dx = -\sqrt{4 - x^2} + C}\]
    
    
    
    \newpage
    
    \begin{tcolorbox}[title={\color{black}\section{Q3}}, colback=white, colframe=\ni, boxrule=1mm, width=1\textwidth]
        3. Obtain the general solution of the equation \( \frac{d^2 y}{dx^2} - 18 \frac{dy}{dx} + 81y = 0 \) 
    \end{tcolorbox}
    
    

    \[ \frac{d^2 y}{dx^2} - 18 \frac{dy}{dx} + 81y = 0 \]    
    We use the ansatz \( y = e^{rx} \), substituting it, along with its corresponding derivatives:
    \[r^2 e^{rx} - 18r e^{rx} + 81 e^{rx} = 0\]
    Dividing through by \( e^{rx} \) (which is never zero), we get the characteristic equation:
    \[ r^2 - 18r + 81 = 0 \]
    We can factorize the quadratic as:
    \[(r - 9)(r - 9) = 0 \]
    This gives us a repeated root:
    \[ r_1 = r_2 = 9 \]
    For a second-order linear homogeneous differential equation with constant coefficients and a repeated root \( r \), the general solution is:
    \[ y(x) = c_1 e^{rx} + c_2 x e^{rx} \]
    Substituting \( r = 9 \) into the general solution, we get:
    \[ y(x) = c_1 e^{9x} + c_2 x e^{9x} \]
    Factorising we get the final solution as:
    \[ \boxed{y(x) = e^{9x} (c_1 + c_2 x)} \]
    \textbf{Interval of validity}: Nothing really stands out, no singularities or undefined behaviours. Therefore, the interval of validity is:
    \[x\in \mathbb{R}\]        
    This means the solution is valid for all real values of \(x\). The constant \(C\) does not effect the solutions validity regardless of its value.
       
    \newpage
    
    \begin{tcolorbox}[title={\color{black}\section{Q4}}, colback=white, colframe=\ni, boxrule=1mm, width=1\textwidth]
        4. Find the particular solution of the differential equation \( \frac{dy}{dx} + 3y x^3 = 0 \), given \( y(0) = 1 \)
    \end{tcolorbox}
    
    \[\frac{dy}{dx} + 3y x^3 = 0\]
    Separate the variables:
    \[\frac{1}{y} \, dy = -3x^3 \, dx\]
    Integrating both sides:
    \[\int \frac{1}{y} \, dy = \int -3x^3 \, dx\]
    \[\ln|y| = -\frac{3x^4}{4} + C\]
    Exponentiating both sides:
    \begin{align*}
        y=&\exp\left(-\frac{3x^4}{4} + C\right)\\
            =&C \exp\left(-\frac{3x^4}{4}\right)
    \end{align*}
    Apply the initial condition \( y(0) = 1 \):
    \[1 = C \exp\left(-\frac{3(0)^4}{4}\right)\]
    \[C = 1\]
    Thus, the particular solution is:
    \[\boxed{y(x) = e^{-\frac{3x^4}{4}}}\]
    \textbf{Interval of validity}: Nothing really stands out, no singularities or undefined behaviours. Therefore, the interval of validity is:
    \[x\in \mathbb{R}\]        
    This means the solution is valid for all real values of \(x\). The constant \(C\) does not effect the solutions validity regardless of its value
    
    \newpage
    
    \begin{tcolorbox}[title={\color{black}\section{Q5}}, colback=white, colframe=\ni, boxrule=1mm, width=1\textwidth]
        5. If \( z = \frac{11 + 10{j}}{9 - 3{j}} \), express both \( \frac{1}{z} \) and \( z + \frac{1}{z} \) in the standard form \( \alpha + \beta {j} \)
    \end{tcolorbox}
    
    \begin{minipage}[t]{0.4\textwidth}
        \[z = \frac{11 + 10j}{9 - 3j}\]    
        \[z = \frac{(11 + 10j)(9 - 3j)}{(9 - 3j)(9 - 3j)}\]
        Simplify the denominator:
        \begin{align*}
            (9 - 3j)(9 + 3j) &= 9^2 - (3j)^2\\ 
            &= 81 - (-9)\\ 
            &= 81 + 9 = 90
         \end{align*}
        Simplify the numerator:
        \begin{align*}
            (11 + 10j)(9 + 3j) &= (11 \cdot 9) + (11 \cdot 3j)\\ &+ (10j \cdot 9) + (10j \cdot 3j)\\[8pt] 
            &= 99 + 33j + 90j + 30j^2\\ 
            &= 99 + 123j + 30(-1)\\ 
            &= 99 + 123j - 30 \\
            &= 69 + 123j   
        \end{align*}
        So:
        \begin{align*}
            z &= \frac{69 + 123j}{90}\\[8pt]
            &= \frac{69}{90} + \frac{123}{90}j\\[8pt]
            &= 0.7\dot{6}+1.3\dot{6} j
        \end{align*}
    \end{minipage}\hfil%
    \begin{minipage}[t]{0.4\textwidth}
        \[\frac{1}{z} = \frac{11 + 10j}{9 - 3j}\]
        \[\frac{1}{z} = \frac{(9 - 3j)(11 - 10j)}{(11 + 10j)(11 - 10j)}\]
        Simplify the denominator:
        \begin{align*}
            (11 + 10j)(11 - 10j) &= 11^2 - (10j)^2\\
            &= 121 - (-100)\\
            &= 121 + 100 = 221
        \end{align*}
        The numerator is the conjugate of the previously calculated numerator, so:        
        \[(9 - 3j)(11 - 10j) = 69 - 123j\]
        Thus:
        \begin{align*}
            \frac{1}{z} &= \frac{69 - 123j}{221}\\[8pt]
            &= \frac{69}{221} + \frac{-123}{221}j\\[8pt]
            &\approx \boxed{0.31 - 0.56j}
        \end{align*}
    \end{minipage}\\
    \vspace{2em}
    now we can calculate $z + \frac{1}{z}$ easily
        \begin{align*}
            z+\frac{1}{z} &= \frac{69}{90} + \frac{123}{90}j + \frac{69}{221} + \frac{-123}{221}j\\[8pt]
            &= \left(\frac{69}{90}+\frac{69}{221}\right) + \left(\frac{123}{90}- \frac{123}{221}\right)j\\[8pt]
            &\boxed{\approx 1.08 + 0.81 j}
    \end{align*}
    in non approximate form this is \(\frac{7153}{6630} +\frac{5371}{6630}j \)
    
    \newpage
    
    \begin{tcolorbox}[title={\color{black}\section{Q6}}, colback=white, colframe=\ni, boxrule=1mm, width=1\textwidth]
        6. Obtain the general solution of \( \frac{d^2 y}{dx^2} - 2 \frac{dy}{dx} - 48y = 5 \)
    \end{tcolorbox}

        \[ \frac{d^2 y}{dx^2} - 2 \frac{dy}{dx} - 48y = 5 \]
        
        Since this is \textbf{non-homogenous} we need to solve for the \textbf{complementary solution (\(\bm{y_c}\))} and then the \textbf{particular solution ($\bm{y_p}$)} the general form of the solution the addition of these $$y(t)=y_c(t)+y_p(t)$$

    \begin{minipage}[t]{0.5\textwidth}
            First lets solve the complementary solution, which solves the associated homogeneous equation:
            \[ \frac{d^2 y}{dx^2} - 2 \frac{dy}{dx} - 48y = 0 \]
            We use the ansatz \( y = e^{rx} \), substituting it, along with its corresponding derivatives:
            \[ r^2 e^{rx} - 2r e^{rx} - 48 e^{rx} = 0 \]
            Dividing through by \( e^{rx} \) (which is never zero), we obtain the characteristic equation:
            \[ r^2 - 2r - 48 = 0 \]
            We can factorize the quadratic as:
            \[(r - 8)(r + 6)=0\]
            This gives us the roots:
            \[ r_1 = 8, \quad r_2 = -6 \]
            The general solution to a second-order linear homogeneous equation is given by:
            \[ y(x) = c_1 e^{r_1x} + c_2 e^{r_2x} \]
            In so for our solution this is
            \[ y_c(x) = c_1 e^{8x} + c_2 e^{-6x} \]
    \end{minipage}\hfil%        
    \begin{minipage}[t]{0.45\textwidth}
        Since the non-homogeneous term is a constant 5, we use the ansatz \(y=A\), where \(A\) is a constant. Since \(A\) is a constant, its derivatives are zero, so:
        \[-48A = 5\]
        \[A=-\frac{5}{48} \]
        Therefore, the particular solution is:
        \[ y_p(x) = -\frac{5}{48} \]

        \hrule
        \vspace{1em}
        
        Now we have complementary solution \((y_c)\) and the particular solution \((y_p)\), The general solution to the non-homogeneous equation can be written as so:
        \[ y(x) = y_h(x) + y_p(x) \]
        plugging in the solved values we have
        \[\boxed{y(x) = c_1 e^{8x} + c_2 e^{-6x} - \frac{5}{48}} \]    
        \textbf{Interval of validity}: Nothing really stands out, no singularities or undefined behaviours. Therefore, the interval of validity is:
        \[
        \begin{tblr}{
                colspec = {l c},
                row{1} = {font=\bfseries}
            }
            \text{Global condition:} & $x \in \mathbb{R}$ \\
            \text{Interval:} & $-\infty < x < \infty$
        \end{tblr}
        \]
        This means the solution is valid for all real values of \(x\). The constant \(C\) does not effect the solutions validity regardless of its value
    \end{minipage}        
    
    
    \newpage
    
    \begin{tcolorbox}[title={\color{black}\section{Q7}}, colback=white, colframe=\ni, boxrule=1mm, width=1\textwidth]
        7. Find \( \frac{\partial z}{\partial x} \) and \( \frac{\partial z}{\partial y} \) when \( z = x y^4 e^{2x} \)
    \end{tcolorbox}
    
    \[z = x y^4 e^{2x}\]
    \[\frac{\partial z}{\partial x} = \frac{\partial\left( x y^4 e^{2x} \right)}{\partial x} \]
    Use the product rule:
    \[\frac{\partial (u(x)\cdot v(x))}{\partial x} = v\frac{\partial u}{\partial x} + u\frac{\partial v}{\partial x}\]
    let $u(x) = x$ and $v(x)=y^4 e^{2x}$
    \[\frac{\partial z}{\partial x}=\frac{\partial\left( x y^4 e^{2x} \right)}{\partial x} = y^4 e^{2x} \frac{\partial (x)}{\partial x} + x \frac{\partial \left( y^4 e^{2x} \right)}{\partial x} \]
    Easy to solve, just treat \(y\) as a constant and differentiate with respect to \(x\):
    \[\frac{\partial (x)}{\partial x}=1, \qquad \frac{\partial \left( y^4 e^{2x} \right)}{\partial x} = 2e^{2x}\]
    now plug in we get
    \begin{align*}
        \frac{\partial z}{\partial x} &= y^4 e^{2x} + x y^4 \left(2e^{2x}\right)\\ 
        &= e^{2x}y^4(2x + 1)
    \end{align*}    
    Thus:
    \[\boxed{\frac{\partial z}{\partial x} = e^{2x}y^4(2x + 1)}\]
    \hrule
    \[z = x y^4 e^{2x}\]
    \[\frac{\partial z}{\partial y} = \frac{\partial \left( x y^4 e^{2x} \right)}{\partial y} \]
    Treat \(x\) as a constant, we only differentiate \( y^4 \):
    \[\frac{\partial z}{\partial y} = \frac{\partial \left( x y^4 e^{2x} \right)}{\partial y} = x e^{2x} \frac{\partial \left(y^4\right) }{\partial y}\]
    \[\frac{\partial \left(y^4\right) }{\partial y} = 4y^3\]    
    Thus:
    \[\boxed{\frac{\partial z}{\partial y} = 4x y^3 e^{2x}}\]
    
    \newpage
    
    \begin{tcolorbox}[title={\color{black}\section{Q8}}, colback=white, colframe=\ni, boxrule=1mm, width=1\textwidth]
        8. Integrate the function \( \int \frac{x + 9}{x(x + 5)} \, dx \)
    \end{tcolorbox}
    
    \[\int \frac{x + 9}{x(x + 5)} \, dx\]
    Split the integral:
    \[\int \frac{x + 9}{x(x + 5)} \, dx = \int \frac{\left(x + 5\right) + 4}{x(x + 5)} = \int \frac{1}{x} \, dx + 4 \int \frac{1}{x(x + 5)} \, dx\]
    The first integral is straightforward:
    \[\int \frac{1}{x} \, dx = \ln|x| + C_1\]
    For the second integral, rewrite:
    \[\int \frac{1}{x(x + 5)} \, dx = \int \frac{1}{x^2 \left( \frac{5}{x} + 1 \right)} \, dx\]
    Now use sub
    \[u = \frac{5}{x} + 1\]
    \[\frac{du}{dx} = -\frac{5}{x^2} \quad \Rightarrow \quad dx = -\frac{x^2}{5} \, du\]
    Substituting into the integral, we get:
    \[\int -\frac{x^2}{5x^2u} \, du = -\frac{1}{5} \int \frac{1}{u} \, du = -\frac{1}{5} \ln|u| + C_2\]
    Substituting \( u = \frac{5}{x} + 1 = \frac{5+x}{x} \) and then combing results yields:
\begin{align*}
    \int \frac{x + 9}{x(x + 5)} \, dx &= \ln|x| + 4 \left( -\frac{1}{5} \ln\left|\frac{x+5}{x}\right|\right)+C\\
    &=\ln|x| -\frac{4}{5} \left( \ln|x + 5| - \ln|x| \right) + C\\ 
    &= -\frac{4 \ln\left|x + 5\right| - 9 \ln\left|x\right|}{5} + C    
\end{align*}
    Thus, the solution is:
    \[\boxed{\int \frac{x + 9}{x(x + 5)} \, dx =  \frac{9 \ln\left|x\right|-4 \ln\left|x + 5\right|}{5} + C}\]
    
    \newpage
    
    \begin{tcolorbox}[title={\color{black}\section{Q9}}, colback=white, colframe=\ni, boxrule=1mm, width=1\textwidth]
        9. Solve the equation \( \frac{dy}{dt} + y \cot t = 5 \sin t \)
    \end{tcolorbox}    
    \[\frac{dy}{dt} + y \cot t = 5 \sin t\]
    \begin{minipage}[t]{0.45\textwidth}
        Use integrating factor \( \mu(t) \):
        \[\mu(t) = e^{\int \cot t \, dt}\]
        Since this is a conventional integral, I shouldn't actually include it in the calculation because it should be widely recognised as \(\ln|\sin t \,|\). Regardless, I'll demonstrate how to solve it quickly:        
        \[\int \cot t \, dt = \int \frac{1}{\tan t} \, dt = \int \frac{\cos t}{\sin t} \, dt\]
        Now use sub
        \[u = \sin t\]
        \[\frac{du}{dt} = \cos t \quad \Rightarrow \quad dx = \frac{1}{\cos t} \, dt\]
        \[\int \frac{\cos t}{\sin t} \, dt = \int \frac{1}{u} du = \ln|\sin t\,|+C\]
        so the int factor \(\mu(t)\) is
        \[\mu(t) =e^{\ln |\sin t\,|} = |\sin t\,|\]
        Multiply both sides of the differential equation by \( \mu(t) \)
    \end{minipage}\hfil%
    \begin{minipage}[t]{0.45\textwidth}
    we get:
    \[\frac{dy}{dt}\sin t  + y \sin t \cot t = 5 \sin^2 t\]
    \[\frac{dy}{dt}\sin t  + y\cos t = 5 \sin^2 t\]
    The left side of the equation is nothing more than the product rule:
    \[\frac{d}{dt} (y \sin t) = 5 \sin^2 t\]
    Integrating both sides, we obtain:
    \begin{align*}
        y \sin t &= \int 5 \sin^2 t \, dt\\
        &= \int 5 \cdot \frac{1 - \cos 2t}{2} \, dt\\ 
        &= \frac{5}{2} \left( \int 1 \, dt - \int \cos 2t \, dt \right)\\
        &= \frac{5}{2} \left( t - \frac{\sin 2t}{2} \right) + C
    \end{align*}
    \end{minipage}\\
    In so \(y\) as an explicit solution is:
    \[y = \frac{5}{2 \sin t} \left( t - \frac{\sin 2t}{2} \right) + \frac{C}{\sin t}\]
    After some rewriting, the final solution is:
    \[ \boxed{ = \frac{1}{4} \csc(t) \left( C + 10t - 5 \sin(2t) \right) } \]
    
    \textbf{Interval of validity}: The function \( \csc(t) \) is undefined where \( \sin(t) = 0 \), which occurs at \( t = n\pi \), where \( n \) is any integer. Therefore, the interval of validity of the solution is:
    \[ t \in \mathbb{R} : \frac{t}{\pi} \notin \mathbb{Z} \]
    This means the solution is valid for all \( t \) being real, except for those values where \( \frac{t}{\pi} \) is an integer. Note for the the interval or global condition in this case, it is not dependant on the constant \(C\).
    
    
    \newpage
    
    \begin{tcolorbox}[title={\color{black}\section{Q10}}, colback=white, colframe=\ni, boxrule=1mm, width=1\textwidth]
        10. Evaluate the integral \( \int_{1}^{3} 4x e^{4x} \, dx \)
    \end{tcolorbox}
       
    \[\int_{1}^{3} 4x e^{4x} \, dx\]
    let \(u=4x \quad dx=\frac{1}{4}\)
    \[\frac{1}{4}\int_{4}^{12} u e^{u} \, dx\]
%    \[\frac{1}{4} \left([u e^{u}]_{4}^{12} - \int_{4}^{12} e^{u}\right)\]
    \[\frac{1}{4} \left([u e^{u}-e^{u}]_{4}^{12}\right)\]
    \[\frac{1}{4} \left(\left(12 e^{12}-e^{12}\right)-\left(4 e^{4}-e^{4}\right)\right)\]
    \[\boxed{\int_{1}^{3} 4x e^{4x} \, dx=\frac{11e^{12}-3e^{4}}{4} \approx 4.4753\times10^5}\]
\end{document}
