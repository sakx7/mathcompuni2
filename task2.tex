\documentclass[a4paper, 12pt]{report}
\usepackage[utf8]{inputenc}
\usepackage{tikz}
\usepackage{graphicx}
\usepackage{geometry}
\usetikzlibrary{calc,patterns,angles,quotes}
\geometry{margin=1in,left=0.6in,right=0.6in,bottom=0.5in}
\usepackage{fancyhdr}
\usepackage{hyperref}
\usepackage{chemfig}
\usetikzlibrary{calc}
\usepackage{enumitem}
\usepackage{placeins}
\usepackage{caption}
\usepackage{array}
\usepackage{float}
\usepackage{amsmath, amssymb, amscd, MnSymbol,mathrsfs}
\usepackage{tcolorbox}
\usepackage{bibref}
\usepackage{bm}
\newcommand{\vect}[1]{\boldsymbol{\mathbf{#1}}}
\usepackage{pdfpages}

\usepackage{empheq}
\usepackage{pgfplots}
\pgfplotsset{compat=1.18}
\usetikzlibrary{calc, angles, quotes}
\usepackage[oldvoltagedirection]{circuitikz}
\usepackage{booktabs}
\usepackage{array}
\usepackage{arydshln}
\usepackage{tcolorbox}

\usepackage{hyperref}
\usepackage{cancel}
\usepackage{changepage}
\usepackage{placeins}
\usepackage{enumitem}
\usepackage{pgf}
\usetikzlibrary{decorations.markings}
\usetikzlibrary{decorations.pathmorphing}
\usepackage{tikz-3dplot}

\renewcommand{\thechapter}{\Alph{chapter}} % Use letters for parts
\renewcommand{\chaptername}{Part} % Change "Chapter" to "Part"
\renewcommand{\thesection}{\Alph{chapter}.\arabic{section}} % Use letters for sections

\usepackage{listings}

\usepackage{minted}

\usepackage{courier}
\usepackage{xcolor}
\usepackage{color}
\def\ni{green!60!black!40!white}

\lstdefinestyle{mypythonstyle}{
    language=Python,
    basicstyle=\footnotesize\ttfamily,
    numbers=left,
    numberstyle=\tiny,
    numbersep=9pt,
    showstringspaces=false,
    frame=single,
    breaklines=true,
    keywordstyle=\color{blue!80!black}\bfseries,  % Darker blue and bold for keywords
    commentstyle=\color{green!60!black},  % Softer gray for comments
    stringstyle=\color{orange},  % Different color for strings
    backgroundcolor=\color{gray!5},  % Light gray for background
    tabsize=4,
    belowcaptionskip=10pt, % Adjust space below caption
    morecomment=[l]{//},  % For inline comments
    literate={~} {$\sim$}{1},  % Converts tilde to a symbol
}

\lstdefinestyle{matlab}{
    language=MATLAB,
    basicstyle=\ttfamily\footnotesize,
    breaklines=true,
    commentstyle=\color{green!60!black},
    keywordstyle=\color{blue!80!black}\bfseries,
    stringstyle=\color{red!70!black},
    numberstyle=\tiny\color{gray},
    frame=single,
    framesep=5pt,
    rulecolor=\color{black!30},
    backgroundcolor=\color{white!94!gray},
    emphstyle=\color{purple}\bfseries,
    emph={[2]\%},
    emphstyle={[2]\color{green!60!black}},
    showstringspaces=false,
    tabsize=4,
    belowcaptionskip=10pt, % Adjust space below caption
    morekeywords={import,classdef,properties,methods,end,
        function,return,if,else,elseif,switch,case,otherwise,
        for,while,break,continue,try,catch,global,persistent},
    morecomment=[l][\color{magenta!70!black}]{\%\%},
    morestring=[b]',
    literate=
    *{0}{{{\color{cyan!70!black}0}}}1
    {1}{{{\color{cyan!70!black}1}}}1
    {2}{{{\color{cyan!70!black}2}}}1
    {3}{{{\color{cyan!70!black}3}}}1
    {4}{{{\color{cyan!70!black}4}}}1
    {5}{{{\color{cyan!70!black}5}}}1
    {6}{{{\color{cyan!70!black}6}}}1
    {7}{{{\color{cyan!70!black}7}}}1
    {8}{{{\color{cyan!70!black}8}}}1
    {9}{{{\color{cyan!70!black}9}}}1
}

\def\link{blue!50!black}
\makeatletter
\renewcommand*\env@matrix[1][\arraystretch]{%
    \edef\arraystretch{#1}%
    \hskip -\arraycolsep
    \let\@ifnextchar\new@ifnextchar
    \array{*{\c@MaxMatrixCols}c}
}
\makeatother


\pagestyle{fancy}
\fancyhf{}
\fancyhead[L]{Module: EG4017 - Engineering Mathematics}
\fancyhead[R]{Page \thepage}

\title{\vspace{3em} \Huge \textbf{Engineering\\ Mathematics and Computing}\\ \vspace{1em} \Large Task 2: Coursework Assessment}
\author{Student Name: Sakariye Abiikar\\ KID: 2371673}
\date{Last Updated: November 6, 2024\\ Submission Deadline: December 12, 2024, 5pm \\[1em] Git Repo : \color{blue}\url{https://github.com/sakx7/mathcompuni2}}

\begin{document}
    
    \maketitle
    \thispagestyle{empty}
    
    \newpage
    \thispagestyle{empty}
    \newgeometry{margin=1in,left=0.6in,right=0.6in,bottom=0in}
    
    \chapter{Mathematics}
    
    \newpage\centering\restoregeometry
    
    \setcounter{page}{1}
    \fancyhead[L]{Part A: Mathematics}
    
    
    \begin{tcolorbox}[title={\color{black}\section{Q1}}, colback=white, colframe=\ni, boxrule=1mm, width=1\textwidth]
        1. Find \( \int \frac{1}{7x + 6} \, dx \)
    \end{tcolorbox}
    To evaluate the integral 
    \[\int \frac{1}{7x + 6} \, dx\]
    use substitution, Let 
    \[u = 7x + 6\]
    then 
    \[\frac{du}{dx} = 7 \quad \Rightarrow \quad dx = \frac{du}{7}\]
    Substituting into the integral, we get
    \[\int \frac{1}{7x + 6} \, dx = \frac{1}{7} \int \frac{1}{u} \, du\]
    this is easy
    \[\frac{1}{7} \int \frac{1}{u} \, du = \frac{1}{7} \ln |u| + C\]
    Finally, substituting back \( u = 7x + 6 \), we get
    \[\boxed{\int \frac{1}{7x + 6} \, dx = \frac{1}{7} \ln |7x + 6| + C}\]
    
    
    \newpage
    
    \begin{tcolorbox}[title={\color{black}\section{Q2}}, colback=white, colframe=\ni, boxrule=1mm, width=1\textwidth]
        2. Find \( \int \frac{x}{\sqrt{4 - x^2}} \, dx \)
    \end{tcolorbox}
    
    To evaluate the integral 
    \[\int \frac{x}{\sqrt{4 - x^2}} \, dx\]
    we can use a substitution method.\\ Let\\[-1.5em] 
    \[u = 4 - x^2\]
    then 
    \[\frac{du}{dx} = -2x \quad \Rightarrow \quad dx = -\frac{du}{2x}\]
    Substituting into integral, we get
    \[\int \frac{x}{\sqrt{4 - x^2}} \, dx = \int \frac{x}{\sqrt{u}} \cdot \left(-\frac{du}{2x}\right) = -\frac{1}{2} \int \frac{1}{\sqrt{u}} \, du\]
    this is easy
    \[-\frac{1}{2} \int u^{-\frac{1}{2}} \, du = -\frac{1}{2} \cdot 2 \sqrt{u} + C = -\sqrt{u} + C\]
    Finally, substituting back \( u = 4 - x^2 \), we get
    \[\boxed{\int \frac{x}{\sqrt{4 - x^2}} \, dx = -\sqrt{4 - x^2} + C}\]
    
    
    
    \newpage
    
    \begin{tcolorbox}[title={\color{black}\section{Q3}}, colback=white, colframe=\ni, boxrule=1mm, width=1\textwidth]
        3. Obtain the general solution of the equation \( \frac{d^2 y}{dx^2} - 18 \frac{dy}{dx} + 81y = 0 \) 
    \end{tcolorbox}
    
    

    \[ \frac{d^2 y}{dx^2} - 18 \frac{dy}{dx} + 81y = 0 \]    
    We use the ansatz \( y = e^{rx} \), with the respective derivative:
    \[r^2 e^{rx} - 18r e^{rx} + 81 e^{rx} = 0\]
    Dividing through by \( e^{rx} \) (which is never zero), we get the characteristic equation:
    \[ r^2 - 18r + 81 = 0 \]
    solve this quadratic equation, relatively easy:
    \[ r^2 - 18r + 81 = (r - 9)(r - 9) = 0 \]
    This gives us a repeated root:
    \[ r_1 = r_2 = 9 \]
    For a second-order linear homogeneous differential equation with constant coefficients and a repeated root \( r \), the general solution is:
    \[ y(x) = c_1 e^{rx} + c_2 x e^{rx} \]
    Substituting \( r = 9 \) into the general solution, we get:
    \[ y(x) = c_1 e^{9x} + c_2 x e^{9x} \]
    So the correct form of the solution is:
    \[ \boxed{y(x) = e^{9x} (c_1 + c_2 x)} \]    
    \newpage
    
    \begin{tcolorbox}[title={\color{black}\section{Q4}}, colback=white, colframe=\ni, boxrule=1mm, width=1\textwidth]
        4. Find the particular solution of the differential equation \( \frac{dy}{dx} + 3y x^3 = 0 \), given \( y(0) = 1 \)
    \end{tcolorbox}
    
    \[\frac{dy}{dx} + 3y x^3 = 0\]
    Separate the variables:
    \[\frac{1}{y} \, dy = -3x^3 \, dx\]
    Integrating both sides:
    \[\int \frac{1}{y} \, dy = \int -3x^3 \, dx\]
    \[\ln|y| = -\frac{3x^4}{4} + C\]
    Exponentiating both sides:
    \begin{align*}
        y=&\exp\left(-\frac{3x^4}{4} + C\right)\\
            =&C \exp\left(-\frac{3x^4}{4}\right)
    \end{align*}
    Apply the initial condition \( y(0) = 1 \):
    \[1 = C \exp\left(-\frac{3(0)^4}{4}\right)\]
    \[C = 1\]
    Thus, the particular solution is:
    \[\boxed{y(x) = e^{-\frac{3x^4}{4}}}\]
        
    
    \newpage
    
    \begin{tcolorbox}[title={\color{black}\section{Q5}}, colback=white, colframe=\ni, boxrule=1mm, width=1\textwidth]
        5. If \( z = \frac{11 + 10{j}}{9 - 3{j}} \), express both \( \frac{1}{z} \) and \( z + \frac{1}{z} \) in the standard form \( \alpha + \beta {j} \)
    \end{tcolorbox}
    
    \[z = \frac{11 + 10{j}}{9 - 3 {j}} \quad\Rightarrow\quad \frac{1}{z} = \frac{9 - 3 {j}}{11 + 10{j}}\]
    rationalize the denominator:
    \[\frac{1}{z} = \frac{9 - 3 {j}}{11 + 10{j}} = \frac{(9 - 3 {j})(11 - 10 {j})}{(11 + 10 {j})(11 - 10 {j})}\]
    First, simplify the denominator:
    \[(11 + 10 {j})(11 - 10 {j}) = 11^2 - (10 {j})^2 = 121 - (-100) = 121 + 100 = 221\]
    Now simplify the numerator:
    \[(9 - 3 {j})(11 - 10 {j}) = (9\cdot11) + (9\cdot-10{j}) + (-3{j}\cdot 11) + (-3 {j}\cdot-10{j})\]
    \[= 99 - 90 {j} - 33 {j}-30 = 69-123 {j}\]
    Thus:
    \[\frac{1}{z} = \frac{69-123 {j}}{221}\]
    Now, break it down:
    \[\boxed{\frac{1}{z} = \frac{69}{221} + \frac{-123}{221} {j} \approx 0.31 - 0.56 {j}}\]

    Now, calculate \( z + \frac{1}{z} \):
    First, let's simplify \( z \):
    \[z = \frac{11 + 10j}{9 - 3j} \cdot \frac{9 + 3j}{9 + 3j} = \frac{(11 + 10j)(9 + 3j)}{(9 - 3j)(9 + 3j)}\]
    Calculate the denominator:
    \[(9 - 3j)(9 + 3j) = 9^2 - (3j)^2 = 81 + 9 = 90\]
    Calculate the numerator:
    \[(11 + 10j)(9 + 3j) = (11 \cdot 9) + (11 \cdot 3j) + (10j \cdot 9) + (10j \cdot 3j) = 99 + 33j + 90j + 30j^2\]
    \[= 99 + 123j + 30(-1) = 99 + 123j - 30 = 69 + 123j\]
    So:
    \[z = \frac{69 + 123j}{90} = \frac{69}{90} + \frac{123}{90}j = \frac{23}{30} + \frac{41}{30}j\]
    \[z + \frac{1}{z} = \left( \frac{23}{30} + \frac{41}{30}j \right) + \left( \frac{69}{221} - \frac{123}{221}j \right)\]
    Find a common denominator for the fractions:
    \[z + \frac{1}{z} = \frac{23 \cdot 221 + 69 \cdot 30}{30 \cdot 221} + \frac{41 \cdot 221 - 123 \cdot 30}{30 \cdot 221}j\]
    \[= \frac{7153}{6630} + \frac{5371}{6630}j\]
    Thus:
    \[\boxed{z + \frac{1}{z} = \frac{7153}{6630} + \frac{5371}{6630}j \approx 1.08 + 0.81j}\]


    \newpage
    
    \begin{tcolorbox}[title={\color{black}\section{Q6}}, colback=white, colframe=\ni, boxrule=1mm, width=1\textwidth]
        6. Obtain the general solution of \( \frac{d^2 y}{dx^2} - 2 \frac{dy}{dx} - 48y = 5 \)
    \end{tcolorbox}
        To obtain the general solution of the non-homogeneous differential equation:
        \[ \frac{d^2 y}{dx^2} - 2 \frac{dy}{dx} - 48y = 5 \]
        
        we follow these steps:
        
        \[ \frac{d^2 y}{dx^2} - 2 \frac{dy}{dx} - 48y = 0 \]
        
        We use the ansatz \( y = e^{rx} \). Substituting this into the homogeneous equation, we get:
        \[ r^2 e^{rx} - 2r e^{rx} - 48 e^{rx} = 0 \]
        
        Dividing through by \( e^{rx} \) (which is never zero), we obtain the characteristic equation:
        \[ r^2 - 2r - 48 = 0 \]
        
        Solving this quadratic equation:
        \[ r^2 - 2r - 48 = (r - 8)(r + 6) = 0 \]
        
        This gives us the roots:
        \[ r_1 = 8, \quad r_2 = -6 \]
        
        Therefore, the general solution to the homogeneous equation is:
        \[ y_h(x) = c_1 e^{8x} + c_2 e^{-6x} \]
        
        Since the non-homogeneous term is a constant (5), we use the ansatz \( y_p = A \), where \( A \) is a constant.
        
        Substituting \( y_p = A \) into the non-homogeneous equation:
        \[ \frac{d^2 A}{dx^2} - 2 \frac{dA}{dx} - 48A = 5 \]
        
        Since \( A \) is a constant, its derivatives are zero:
        \[ -48A = 5 \]
        
        Solving for \( A \):
        \[ A = -\frac{5}{48} \]
        
        Therefore, the particular solution is:
        \[ y_p(x) = -\frac{5}{48} \]
        
        The general solution to the non-homogeneous equation is the sum of the homogeneous solution and the particular solution:
        \[ y(x) = y_h(x) + y_p(x) \]
        \[ y(x) = c_1 e^{8x} + c_2 e^{-6x} - \frac{5}{48} \]    
    \newpage
    
    \begin{tcolorbox}[title={\color{black}\section{Q7}}, colback=white, colframe=\ni, boxrule=1mm, width=1\textwidth]
        7. Find \( \frac{\partial z}{\partial x} \) and \( \frac{\partial z}{\partial y} \) when \( z = x y^4 e^{2x} \)
    \end{tcolorbox}
    
    To find \( \frac{\partial z}{\partial x} \), treat \( y \) as a constant:
    \[z = x y^4 e^{2x}\]
    Differentiate with respect to \( x \):
    \[\frac{\partial z}{\partial x} = \frac{\partial}{\partial x} \left( x y^4 e^{2x} \right)\]
    Use the product rule. For the terms \( x \) and \( y^4 e^{2x} \), we get:
    \[\frac{\partial z}{\partial x} = \left( \frac{\partial}{\partial x} x \right) y^4 e^{2x} + x \frac{\partial}{\partial x} \left( y^4 e^{2x} \right)\]
    The derivative of \( x \) with respect to \( x \) is 1, and the derivative of \( y^4 e^{2x} \) with respect to \( x \) is \( y^4 \cdot 2e^{2x} \). Therefore:
    \[\frac{\partial z}{\partial x} = y^4 e^{2x} + x y^4 \cdot 2e^{2x} = e^{2x}(2x + 1) y^4\]
    Thus:
    \[\boxed{\frac{\partial z}{\partial x} = e^{2x} (2x + 1) y^4}\]
    To find \( \frac{\partial z}{\partial y} \), treat \( x \) as a constant:
    \[z = x y^4 e^{2x}\]
    Differentiate with respect to \( y \):
    \[\frac{\partial z}{\partial y} = \frac{\partial}{\partial y} \left( x y^4 e^{2x} \right)\]
    Since \( x \) and \( e^{2x} \) are constants with respect to \( y \), we only differentiate \( y^4 \):
    \[\frac{\partial z}{\partial y} = x e^{2x} \cdot \frac{\partial}{\partial y} y^4 = x e^{2x} \cdot 4y^3\]
    Thus:
    \[\boxed{\frac{\partial z}{\partial y} = 4x y^3 e^{2x}}\]
    
    \newpage
    
    \begin{tcolorbox}[title={\color{black}\section{Q8}}, colback=white, colframe=\ni, boxrule=1mm, width=1\textwidth]
        8. Integrate the function \( \int \frac{x + 9}{x(x + 5)} \, dx \)
    \end{tcolorbox}
    
    \[\int \frac{x + 9}{x(x + 5)} \, dx\]
    split the integral:
    \[\int \frac{x + 9}{x(x + 5)} \, dx = \int \frac{1}{x} \, dx + 4 \int \frac{1}{x(x + 5)} \, dx\]
    The first integral is straightforward:
    \[\int \frac{1}{x} \, dx = \ln|x| + C_1\]
    For the second integral, rewrite:
    \[\int \frac{1}{x(x + 5)} \, dx = \int \frac{1}{x^2 \left( \frac{5}{x} + 1 \right)} \, dx\]
    Let \( u = \frac{5}{x} + 1 \), then \( dx = -\frac{x^2}{5} \, du \).
    Substituting into the integral, we get:
    \[-\frac{1}{5} \int \frac{1}{u} \, du = -\frac{1}{5} \ln|u| + C_2\]
    Combing results and Substituting back \( u = \frac{5}{x} + 1 \):
    \[\ln|x| + 4 \left( -\frac{1}{5} \ln|x + 5| + \frac{1}{5} \ln|x| \right) + C = -\frac{4 \ln\left|x + 5\right| - 9 \ln\left|x\right|}{5} + C\]
    Thus, the solution is:
    \[\boxed{\int \frac{x + 9}{x(x + 5)} \, dx =  -\frac{4 \ln\left|x + 5\right| - 9 \ln\left|x\right|}{5} + C}\]
    
    \newpage
    
    \begin{tcolorbox}[title={\color{black}\section{Q9}}, colback=white, colframe=\ni, boxrule=1mm, width=1\textwidth]
        9. Solve the equation \( \frac{dy}{dt} + y \cot t = 5 \sin t \)
    \end{tcolorbox}
    
    \[\frac{dy}{dt} + y \cot t = 5 \sin t\]
    First, we identify the integrating factor \( \mu(t) \):
    \[\mu(t) = e^{\int \cot t \, dt} = e^{\ln |\sin t|} = |\sin t|\]
    Multiplying both sides of the differential equation by \( \mu(t) \), we get:
    \[\sin t \frac{dy}{dt} + y \sin t \cot t = 5 \sin^2 t\]
    \[\frac{dy}{dt} \sin t  + y\cos(t) = 5 \sin^2 t\]
    \[u'v+uv'\]
    using the product rule, the derivative of $y\sin(t)$ with respect to $t$ is the left hand side.
    \[\frac{d}{dt} (y \sin t) = 5 \sin^2 t\]
    Integrating both sides, we obtain:
    \[y \sin t = \int 5 \sin^2 t \, dt\]
    Using the identity \( \sin^2 t = \frac{1 - \cos 2t}{2} \), we get:
    \[y \sin t = \int 5 \cdot \frac{1 - \cos 2t}{2} \, dt = \frac{5}{2} \left( \int 1 \, dt - \int \cos 2t \, dt \right)\]
    \[= \frac{5}{2} \left( t - \frac{\sin 2t}{2} \right) + C\]
    Solving for \( y \), we get:
    \[y = \frac{5}{2 \sin t} \left( t - \frac{\sin 2t}{2} \right) + \frac{C}{\sin t}\]
    Thus, the solution is:
    \[\boxed{y = C \csc(t) + \frac{5}{2} (t\csc(t)-\cos(t))}\]

    
    \newpage
    
    \begin{tcolorbox}[title={\color{black}\section{Q10}}, colback=white, colframe=\ni, boxrule=1mm, width=1\textwidth]
        10. Evaluate the integral \( \int_{1}^{3} 4x e^{4x} \, dx \)
    \end{tcolorbox}
       
    \[\int_{1}^{3} 4x e^{4x} \, dx\]
    let \(u=4x \quad dx=\frac{1}{4}\)
    \[\frac{1}{4}\int_{4}^{12} u e^{u} \, dx\]
%    \[\frac{1}{4} \left([u e^{u}]_{4}^{12} - \int_{4}^{12} e^{u}\right)\]
    \[\frac{1}{4} \left([u e^{u}-e^{u}]_{4}^{12}\right)\]
    \[\frac{1}{4} \left(\left(12 e^{12}-e^{12}\right)-\left(4 e^{4}-e^{4}\right)\right)\]
    \[\boxed{\int_{1}^{3} 4x e^{4x} \, dx=\frac{11e^{12}-3e^{4}}{4} \approx 4.4753\times10^5}\]
\end{document}
